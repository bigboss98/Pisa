\chapter{Data Visualization}
For preparing data for data mining task it is essential to have an overall picture of your data, so we have to
gain insight in your data, with respect to your project goals and should be general to understand properties.\newline
You should discover semantics of data and also to discover statistical charecteristics of your data in
order to have a more understanding of data.

Data can be represent usually by $3$ modes:
\begin{itemize}
    \item Record: we have a matrix representation that will represent data and it is divided in:
            \begin{itemize}
                \item Data Matrix: If data objects have the same fixed set of numeric attributes,
                      then the data objects can be thought of as points in a multidimensional space,
                      where each dimension represents a distinct attribute.\newline
                      Such data set can be represented by an $m$ by $n$ matrix, where there are $m$ rows,
                      one for each object, and $n$ columns, one for each attribute,
                      as you can see in figure \ref{img:dataMatrix}.
                \item Document Matrix: each document becomes a ‘term’ vector, where each term 
                      is a component (attribute) of the vector and the value of each component is 
                      the number of times the corresponding term occurs in the document.
                \item Transition Data: A special type of record data, where each record (transaction)
                      involves a set of items, so for example consider a grocery store.\newline
                      The set of products purchased by a customer during one shopping trip constitute a transaction,
                      while the individual products that were purchased are the items,
                      as it possible to note in figure \ref{img:transiction}.
            \end{itemize}
    \item Graph
    \item Order
\end{itemize}
Data is a collection of data objects and their attributes, where the last one is a property or 
characteristic of an object like eye color of a person and a data object is a collection of 
attributes that descrive an object.

There are different types of attributes:
\begin{description}
    \item [Nominal/Categorical: ] attribute values in a finite domain, categories, “name of things” like
                                  eye color, zip codes.
    \item [Binary: ] nominal attribute with only $2$ states ($0$ and $1$) where we have \emph{symmetric binary},
                     in which both outcomes are equally important, or also \emph{asymmetric binary} where 
                     outcomes are not equally important, like medical test positive vs negative, and the convention
                     is to assign $1$ to most important outcome.
    \item [Ordinal: ] finite domain with a meangniful ordering on the domain, like rankings, grades and height.
    \item [Numeric: ] quantity (integer or real-valued) measured on a scale of equal-sized units and which
                      values have order, example of this data are temperatures in Celsius and calendar dates.
    \item [Ratio-scaled: ] we can speak of values as being an order of magnitude larger than the unit of measurement
                           and example are length, elapsed time and so on.
\end{description}
There is also a distinction about which data an attribute can have:
\begin{description}
    \item [Discrete Attribute: ] has only a finite or countably infinite set of values and often are represented
                                 as integer variables, where we can note that binary attributes are a special
                                 case of discrete attributes.
    \item [Continuous Attribute: ] has real numbers as attribute values and practically real values can only
                                   be measured and represented using a finite number of digits.\newline
                                   Examples are temperature, weight, or height and also continuous attributes
                                   are typically represented as floating-point variables.
\end{description}
The type of an attribute depends on which of the following properties/operations it possesses:
\begin{description}
    \item [Distinctness: ] $= \neq$
    \item [Order: ] $< >$
    \item [Differences: ] are $+ -$
    \item [Ratios: ] are $* /$
\end{description}
We have that Nominal attribute has only distinctness, ordinal attribute add the order property, interval
attribute has also differences and in the end ratio attribute has all $4$ properties.

 Poor data quality negatively affects many data processing efforts
“The most important point is that poor data quality is an unfolding
disaster.
– Poor data quality costs the typical company at least ten percent
(10%) of revenue; twenty percent (20%) is probably a better
estimate.”

Some \emph{Data quality} issues are the following:
\begin{description}
•••••Syntactic accuracy: Entry is not in the domain.
– Examples: fmale in gender, text in numerical a8ributes, ... Can be
checked quite easy.
Semantic accuracy: Entry is in the domain but not correct
– Example: John Smith is female
– Needs more informa4on to be checked (e.g. “business rules”).
Completeness: is violated if an entry is not correct although
it belongs to the domain of the attribute.
– Example: Complete records are missing, the data is biased (A bank has
rejected customers with low income.)
Unbalanced data: The data set might be biased extremely
to one type of records.
– Example: Defec4ve goods are a very small frac4on of all.
Timeliness: Is the available data up to date?
\end{description}
Data set may include data objects that are duplicates, or almost duplicates of one another and that is 
a major issue when we are merging data from heterogeneous sources, an example can be a same person with
multiple email addresses.
\emph{Data cleaning} is the process of dealing with duplicate data issues 
When should duplicate data not be removed?

Figure: In figure we change from Lira to Euro and that explain a dramatic reduction on figure and that is 
a problem so we have to rescale values that makes consistent data.

Figure: 

Distribution of data observation is important to recognize if we have symmetric data, skewed data, bimodal
pattern (which usually shows subpopulation on attribute)
